\documentclass[a4paper,11pt,notitlepage,fullpage]{article}
%\documentclass{report}

\usepackage{fullpage}
\usepackage[utf8]{inputenc}
%\usepackage[ngerman]{babel}
\usepackage[english]{babel}
\usepackage{amsmath}
\usepackage{amssymb}
\usepackage{latexsym}
\usepackage{mathtools}
\usepackage{listings}
\usepackage{algorithm}
\usepackage{algpseudocode}
\usepackage{graphicx}
\usepackage{booktabs}
\usepackage{hhline}
\usepackage{amsthm}
\usepackage{cite}
\usepackage{wrapfig}
\usepackage{hyperref}
\usepackage{titling}
\usepackage{color}

\setlength{\droptitle}{-60pt}

\DeclareMathOperator{\arccosh}{arccosh}
\DeclareMathOperator{\sgn}{sgn}
\DeclareMathOperator{\sech}{sech}




\begin{document}
\author{Florian Bogner}
\title{Differential Geometry - Exercise 3}
\maketitle

\begin{enumerate}

\item \emph{Compute the evolute of the cycloid $\gamma(t) = (t-\sin t, 1-\cos t)$. Show that the evolute is a congruent copy of the cycloid.}

Recall that the evolute is defined as $\mathcal E(t) = \gamma(t) + \frac{1}{\kappa_s(t)}\nu_s(t)$. From the first exercise we know that 
\begin{equation*}
\kappa_s(t) = - \frac{1}{\sqrt{8}} \frac{1}{\sqrt{1 - \cos t}}. 
\end{equation*}
We can compute $\nu_s(t)$ as follows:
\begin{equation*}
\nu_s(t) = \frac{\dot\gamma(t)^\perp}{\left\|\dot\gamma\right\|} = \frac{(-\sin t, -\cos t + 1)}{\sqrt{2}\sqrt{1-\cos t}}
\end{equation*}
Thus (with $s:= t - \pi$):
\begin{align*}
\mathcal E(t) &= \gamma(t) + 2 (\sin t, \cos t - 1) \\
&= (t + \sin t, - 1 + \cos t) \\
&= (s + \pi - \sin s, - 1 - \cos s) \\
&= (s - \sin s + \pi, 1 - \cos s - 2) \\
&= \gamma(s) + (\pi, -2)
\end{align*}
The evolute of the cycloid is therefore the cycloid translated by $(\pi, -2)$. \qed


\item \emph{Consider the curve $\gamma(t) = (t, \cosh t)$ (the \emph{catenary}). Show that the involute of $\gamma$ with the starting point $\gamma(0)$ is an arc of the tractrix. (One of the parametrizations of the tractrix: $(\log\tan\frac{s}{2} + \cos s, \sin s)$}.)

The involute of $\gamma$ is defined as:
\begin{equation*}
\mathcal I_{\gamma}(t) = \gamma(t) - \mathcal L_0^t(\gamma)\frac{\dot\gamma(t)}{\|\dot\gamma(t)\|}
\end{equation*}
The occurring terms are:
\begin{align*}
\dot\gamma(t) &= (1, \sinh t) \\
\|\dot\gamma(t)\| &= \sqrt{1 + \sinh^2 t} = \sqrt{\cosh^2 t} = \cosh t \\
\mathcal L_0^t(\gamma) &= \int_0^t \cosh s ~ds = \sinh t
\end{align*}
Thus:
\begin{align*}
\mathcal I_{\gamma}(t) &= (t, \cosh t) - \sinh t \frac{(1, \sinh t)}{\cosh t} \\
&= \left(t - \tanh t, \cosh t - \frac{\sinh^2 t}{\cosh t}\right) \\
&= \left(t - \tanh t, \frac{\cosh^2 t}{\cosh t} - \frac{\cosh^2 t - 1}{\cosh t}\right) \\
&= \left(t - \tanh t, \frac{1}{\cosh t}\right)
\end{align*}

%Stupid piece of unintelligeble calculations:
%To show this is a reparametrization of the tractrix given in the hint, we take the equality of the second component as a definition and prove the equality of the first components.
%\begin{gather*}
%\frac{1}{\cosh t} = \sin s \\
%t = \arccosh\frac{1}{\sin s}
%\end{gather*}
%Note that $t \in [0, \infty)$ and correspondingly $s \in \left[\frac{\pi}{2}, \pi\right)$. We are using a number of trig identities, namely:
%\begin{align*}
%\arccosh x &= \ln\left(x+\sqrt{x^2-1}\right) \\
%\tanh x &= \frac{\sinh x}{\cosh x} \\
%\sinh\arccosh x &= \sqrt{-1+x^2} \\
%\sqrt{\frac{1}{\sin^2 x}-1} &= \frac{\cos x}{\sin x} \\
%\sin x \cdot \sinh\arccos\frac{1}{\sin x} &= \left|\cos x\right| ~\footnotemark \\
%\tan\frac{x}{2} &= \frac{1-\cos x}{\sin x}
%\end{align*}
%\footnotetext{Since you probably can't find this one in a formula collection and the deadline is fast approaching, take this: \url{https://www.wolframalpha.com/input/?i=plot+sinh\%28arccosh\%281\%2Fsin+s\%29\%29*sin+s+and+cos+s}}
%With this we have:
%\begin{align*}
%t - \tanh t &= \arccos\frac{1}{\sin s} - \tanh\arccos\frac{1}{\sin s} \\
%&= \ln\left(\frac{1}{\sin s}+\sqrt{\frac{1}{\sin^2 s}-1}\right) - \frac{\sinh\arccos\frac{1}{\sin s}}{\frac{1}{\sin s}} \\
%&= \ln\left(\frac{1-\cos s}{\sin s}\right) - \left|\cos s\right| \\
%&= \ln\tan\frac{s}{2} + \cos s
%\end{align*}
%
%There we have it, defining the second component to match also matches the first. The involute of the catenary is indeed the right arc of the tractrix. There has to be a better way though. \qed
%Elegant proof disregarding the hint:
The tractrix is defined as the curve whose tangents meet the $x$-axis in unit distance, so we have to show
\begin{equation*}
\mathcal I_\gamma(t) + 1 \cdot \frac{\dot{\mathcal I}_\gamma(t)}{\|\dot{\mathcal I}_\gamma(t)\|} = (t, 0)
\end{equation*}
Starting from the left side, we calculate the occurring terms:
\begin{align*}
\dot{\mathcal I}_\gamma(t) &= \left(1 - (1 - \tanh^2 t), \frac{-\sinh t}{\cosh^2 t} \right) \\
&= \left( \tanh^2 t, \frac{-\tanh t}{\cosh t} \right) \\
\|\dot{\mathcal I}_\gamma(t)\| &= \sqrt{\tanh^4 t + \frac{\tanh^2 t}{\cosh^2 t}} \\
&= \tanh t \cdot \sqrt{\frac{\sinh^2 t}{\cosh^2 t} + \frac{1}{\cosh^2 t}} \\
&= \tanh t \\
\frac{\dot{\mathcal I}_\gamma(t)}{\|\dot{\mathcal I}_\gamma(t)\|} &= \left(\tanh t, -\frac{1}{\cosh t} \right)
\end{align*}
Therefore:
\begin{equation*}
\mathcal I_\gamma(t) + 1 \cdot \frac{\dot{\mathcal I}_\gamma(t)}{\|\dot{\mathcal I}_\gamma(t)\|} =
\left(t - \tanh t + \tanh t, \frac{1}{\cosh t}-\frac{1}{\cosh t} \right) = (t, 0)
\end{equation*}
Thus, the involute of the catenary is indeed the tractrix. \qed

\item \emph{Let $\gamma$ be a curve with monotonic and nowhere vanishing curvature. Show that involutes of the evolute $\gamma$ are curves parallel to $\gamma$.}

We have the evolute and the involute of the evolute as:
\begin{align*}
\mathcal E_\gamma &= \gamma + \frac{1}{\kappa_s}\nu_s \\
\mathcal I_{\mathcal E_\gamma, t_0} &= \mathcal E_\gamma - \mathcal L_{t_0}^t(\mathcal E_\gamma) \frac{\dot{\mathcal E}_\gamma}{\|\dot{\mathcal E}_\gamma \|}
\end{align*}
Lets calculate the occurring terms (using $\dot\nu_s = -\kappa_s\dot\gamma$):
\begin{align*}
\dot{\mathcal E}_\gamma &= \dot\gamma +\frac{\dot\nu_s(t)\kappa_s - \nu_s\dot\kappa_s}{\kappa_s^2} \\
&= \dot\gamma - \frac{\kappa_s^2 \dot\gamma}{\kappa_s^2} - \frac{\dot\kappa_s\nu_s}{\kappa_s^2} \\
&= - \frac{\dot\kappa_s\nu_s}{\kappa_s^2} \\
\|\dot{\mathcal E}_\gamma\| &=  \left|\frac{\dot\kappa_s}{\kappa_s^2}\right|\cdot\|\nu_s\| = \left|\frac{\dot\kappa_s}{\kappa_s^2}\right|
\end{align*}
Without loss of generality assume that $\kappa_s = \kappa > 0$ and $\dot\kappa > 0$. Mirror and/or reverse the curve if necessary. This is valid because of the monotonic and nowhere vanishing curvature.
\begin{align*}
\mathcal L_{t_0}^t(\mathcal E_\gamma) &= \int_{t_0}^t \frac{\dot\kappa}{\kappa^2} dt = -\frac{1}{\kappa}\Big|_{t_0}^t = \frac{1}{\kappa(t_0)} - \frac{1}{\kappa(t)} \\
\mathcal I_{\mathcal E_\gamma, t_0} &= \gamma + \frac{1}{\kappa}\nu_s - \left(\frac{1}{\kappa(t_0)} - \frac{1}{\kappa(t)}\right) \nu_s \\
&= \gamma + \frac{1}{\kappa(t_0)}\nu_s
\end{align*}

Therefore the involute of the evolute is indeed a parallel curve with the distance being the radius of the osculation circle in the starting point of the evolute. \qed

\item \emph{Prove the Tait-Kneser theorem: the osculating circles of a curve with a monotonic non-vanishing curvature are disjoint and nested. (\emph{Hint:} two circles are disjoint and nested if and only if the distance between their centers is smaller than the difference of their radii.)}

Consider the two osculating circles at $t_1$ and $t_2$. Their centers are connected by the evolute of $\gamma$, whose length has been calculated in the previous example. This length is of course longer than the distance between the centers. Thus we have:
\begin{equation*}
\left\| c_1 - c_2 \right\| < \mathcal L_{t_1}^{t_2}(\mathcal E_\gamma) = \left|\frac{1}{\kappa(t_1)} - \frac{1}{\kappa(t_2)}\right| = \left|r_1 - r_2\right|
\end{equation*}
\qed




\item Oof.


\item Big oof.






















\end{enumerate}

\end{document}