\documentclass[a4paper,11pt,notitlepage,fullpage]{article}
%\documentclass{report}

\usepackage{fullpage}
\usepackage[utf8]{inputenc}
%\usepackage[ngerman]{babel}
\usepackage[english]{babel}
\usepackage{amsmath}
\usepackage{amssymb}
\usepackage{latexsym}
\usepackage{mathtools}
\usepackage{listings}
\usepackage{algorithm}
\usepackage{algpseudocode}
\usepackage{graphicx}
\usepackage{booktabs}
\usepackage{hhline}
\usepackage{amsthm}
\usepackage{cite}
\usepackage{wrapfig}
\usepackage{hyperref}
\usepackage{titling}
\usepackage{color}

\setlength{\droptitle}{-60pt}


\begin{document}
\author{Florian Bogner}
\title{Differential Geometry - Exercise 5}
\maketitle

\begin{enumerate}
\item \emph{Let $\sigma: U \to M$ be a surface patch, and let $\tau = \sigma \circ \phi: V \to M$ be a reparametrization of $\sigma$ by a diffeomorphism $\phi: V \to U$. Further let $\binom{EF}{FG}$ be the matrix of the symmetric bilinear form $\sigma^*(I^M)$. Express the matrix of the symmetric bilinear form $\tau^*(I^M)$ in terms of $\sigma^*(I^M)$ and the Jacobian matrix of $\phi$.}

Let $(a,b) \in V$ and let $\binom{E^*F^*}{F^*G^*}$ denote the matrix of the symmetric bilinear form $\tau^*(I^M)$. Also for ease of notation let the arguments of $\sigma$ and its derivatives be $(u,v) := \phi(a, b) =: (\phi_1, \phi_2)$. Finally, let $J$ be the Jacobian of $\phi$.
\begin{align*}
\tau_a &= (\sigma_u, \sigma_v) \cdot \binom{\phi_{1,a}}{\phi_{2,a}} \\
\tau_b &= (\sigma_u, \sigma_v) \cdot \binom{\phi_{1,b}}{\phi_{2,b}} \\
(\tau_a, \tau_b) &= (\sigma_u, \sigma_v) \cdot J \\
E^* = \langle\tau_a, \tau_a\rangle &= (\phi_{1,a},\phi_{2,a}) \cdot \langle(\sigma_u, \sigma_v),(\sigma_u, \sigma_v)\rangle \cdot \binom{\phi_{1,a}}{\phi_{2,a}} \\
&= (\phi_{1,a},\phi_{2,a}) \cdot \binom{E~F}{F~G} \cdot \binom{\phi_{1,a}}{\phi_{2,a}} \\
F^* = \langle\tau_a, \tau_b\rangle &= (\phi_{1,a},\phi_{2,a}) \cdot \binom{E~F}{F~G} \cdot \binom{\phi_{1,b}}{\phi_{2,b}} \\
G^* = \langle\tau_b, \tau_b\rangle &= (\phi_{1,b},\phi_{2,b}) \cdot \binom{E~F}{F~G} \cdot \binom{\phi_{1,b}}{\phi_{2,b}}
\end{align*}
Finally:
\begin{equation*}
\binom{E^*F^*}{F^*G^*} = J^T \cdot \binom{E~F}{F~G} \cdot J
\end{equation*}
\qed

\item
\begin{enumerate}
\item \emph{Recall that a symmetric bilinear form $\alpha$ is called degenerate if there is
$v$ such that $\alpha(v,w) = 0$ for all $w$. Let $\alpha: W \times W \to \mathbb R$ be a symmetric
bilinear form on $W$, and $F : V \to W$ be a linear map. Show: if
$\ker F \neq \{0\}$, then the form $F^*\alpha$ is degenerate.}

Suppose $u \in \ker F, u \neq 0$. Then, since $\alpha$ is bilinear:
\begin{equation*}
F^*\alpha(u,v) = \alpha(F(u), F(v)) = \alpha(0, F(v)) = 0 ~~~~ \forall v\in W
\end{equation*}
\qed

\item \emph{Let $F : V \to W$ and $G: W \to U$ be linear maps between vector spaces,
and let $\alpha: U \times U \to R$ be a symmetric bilinear form on $U$. Show:}
\begin{equation*}
(G \circ F)^* \alpha = F^*(G^*\alpha).
\end{equation*}

Just use Definition 2.32 three times, lol.
\begin{multline*}
(G \circ F)^*\alpha(u,v) = \alpha((G \circ F)(u), (G \circ F)(v)) =\\= \alpha(G(F(u)), G(F(v))) = G^*\alpha(F(u), F(v)) = F^*(G^*\alpha)(u,v).
\end{multline*}
\qed
\end{enumerate}

\item \emph{The surface path}
\begin{equation*}
\sigma^0(u,v) = (\cosh u \cos v, \cosh u \sin v, u)
\end{equation*}
\emph{parametrizes the catenoid (the surface of revolution obtained from the catenary). The surface patch}
\begin{equation*}
\sigma^1(u,v) = (-\sinh u \sin v, \sinh u \cos v, -v)
\end{equation*}
\emph{parametrizes a helicoid (make sure that you see the rulings). As the domain for both patches take $\mathbb R \times (0,2\pi)$. Put}
\begin{equation*}
\sigma^t(u,v) = \sigma^0(u,v) \cos t + \sigma^1(u,v) \sin t, ~~~ 0 \leq t \leq \frac{\pi}{2}
\end{equation*}
\emph{Show that for all $t$ the map $\sigma^0(u,v) \mapsto \sigma^t(u,v)$ is a path isometry.}

Corollary 2.34 basically tells us to compare the matrices associated with the first fundamental forms. Since $\sigma^t = \sigma^0$ for $t = 0$ it suffices to calculate the one associated with $\sigma^t$ and if it is independent of $t$ the map is indeed a path isometry. Since calculating by hand is boring and error-prone and it is getting late, we ask the Computer Algebra System $sympy$ to do it. We have technology.\footnote{\url{https://www.youtube.com/watch?v=Q6ctb-Pb3lc}} The code is on the last page.
\begin{align*}
E^t = \langle\sigma^t_u, \sigma^t_u\rangle &= \cosh^{2}{\left (u \right )} \\
F^t = \langle\sigma^t_u, \sigma^t_v\rangle &= 0 \\
G^t = \langle\sigma^t_v, \sigma^t_v\rangle &= \cosh^{2}{\left (u \right )}
\end{align*}
It is indeed independent of $t$. \qed

\item \emph{Show that the surface of revolution}
\begin{equation*}
\sigma(u,v) = (f(u) \cos v, f(u) \sin v, g(u))
\end{equation*}
\emph{can be conformally parametrized by $(v,w)$ where $w = w(u)$. \\ Compute the substituition $w$ for the geographic parametrization of the sphere:}
\begin{equation*}
\sigma(u,v) = (\sin u \cos v, \sin u \sin v, \cos v)
\end{equation*}
\emph{Additional question: why is this parametrization useful for Google Maps?}

Theorem 2.45 tells us to look at the first fundamental form matrices. We calculated them last exercise in example 3. To be conformal we need
\begin{equation*}
\left(\begin{matrix}
f'(u)^2 + g'(u)^2 & 0 \\
0 & f(u)^2
\end{matrix}\right) = \lambda(p) \cdot \left(\begin{matrix} 1 & 0 \\ 0 & 1 \end{matrix}\right)
\end{equation*}
Let $u$ be a function $u(w)$. This leads us to the following differential equation for $u(w)$:
\begin{equation*}
\left(f'(u)\cdot u'\right)^2 + \left(g'(u)\cdot u'\right)^2 = f(u)^2
\end{equation*}
or alternativly:
\begin{equation*}
u' = \frac{f\left(u\right)}{\sqrt{f'\left(u\right)^2 + g'\left(u\right)^2}}
\end{equation*}
Since $f > 0$ and smooth and therefore locally Lipschitz and all that jazz this can be solved with a suitable starting value $u(w_0) = u_0$. Since both the enumerator and the denominator and therefore $u'$ is positive, the solution $u$ is monotonically growing and therefore bijective. The inverse function $w(u)$ therefore exists.\footnote{Did you mean to write $u = u(w)$ in the Angabe? This would make more sense to me.}

The parametrization $(v, w) \mapsto (f(u(w)) \cos v, f(u(w)) \sin v, g(u(w)))$ is therefore by construction conformal.

For the sphere the differential equation becomes
\begin{equation*}
u' = \frac{\sin u}{1}
\end{equation*}
The solution is, according to \emph{Wolfram Alpha}
\begin{align*}
u(w) &= 2 \cot^{-1}\left(e^{-w}\right) \\
w(u) &= -\ln\cot\frac{u}{2}
\end{align*}





About Google Maps: This is basically the Mercator projection. As Google Maps is most predominantly used for large scale maps (i.e. zoomed in), you would want a projection that minimizes local distortion, i.e. square cityblocks should look like squares, not oblique rhomboids. But this is directly a consequence of being conform. Another requirement is that north should always point up. With these two requirements the Mercator projection is basically the only one left. 

\item 

\item No. Follows from 7 and the existence of the sphere. \qed

\item


\end{enumerate}

\newpage

\begin{verbatim}
import sympy as sp
from sympy import sin, cos, sinh, cosh, diff

class Vec3:
    def __init__(self, x, y, z):
        self.x = x
        self.y = y
        self.z = z

    def __add__(self, other):
        return Vec3(self.x + other.x,
                    self.y + other.y,
                    self.z + other.z)
        
    def __mul__(self, scalar):
        return Vec3(scalar * self.x,
                    scalar * self.y,
                    scalar * self.z)

    def diff(self, var):
        return Vec3(diff(self.x, var),
                    diff(self.y, var),
                    diff(self.z, var))
        
    def dot(self, other):
        return self.x * other.x + self.y * other.y + self.z * other.z

sp.init_printing(forecolor = 'White')
u, v, t = sp.symbols("u v t")

sig0 = Vec3( cosh(u) * cos(v),
             cosh(u) * sin(v),
             u )
sig1 = Vec3(-sinh(u) * sin(v),
             sinh(u) * cos(v),
            -v )
sigt = sig0 * cos(t) + sig1 * sin(t)

sigt_u = sigt.diff(u)
sigt_v = sigt.diff(v)

E = sigt_u.dot(sigt_u)
F = sigt_u.dot(sigt_v)
G = sigt_v.dot(sigt_v)

print(sp.latex(sp.simplify(E)))
print(sp.latex(sp.simplify(F)))
print(sp.latex(sp.simplify(G)))
\end{verbatim}


\end{document}