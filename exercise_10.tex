\documentclass[a4paper,11pt,notitlepage,fullpage]{article}
%\documentclass{report}

\usepackage{fullpage}
\usepackage[utf8]{inputenc}
%\usepackage[ngerman]{babel}
\usepackage[english]{babel}
\usepackage{amsmath}
\usepackage{amssymb}
\usepackage{latexsym}
\usepackage{mathtools}
\usepackage{listings}
\usepackage{algorithm}
\usepackage{algpseudocode}
\usepackage{graphicx}
\usepackage{booktabs}
\usepackage{hhline}
\usepackage{amsthm}
\usepackage{cite}
\usepackage{wrapfig}
\usepackage{hyperref}
\usepackage{titling}
\usepackage{color}

\setlength{\droptitle}{-60pt}


\begin{document}
\author{Florian Bogner}
\title{Differential Geometry - Exercise 10}
\maketitle


\begin{enumerate}
\item From page 101 in the lecture notes we know that we can identify $X, Y, Z$ with their corresponding derivations and write those as:
\begin{align*}
D_X &= u^1 \frac{\partial}{\partial x^1} + u^2 \frac{\partial}{\partial x^2} \\
D_Y &= v^1 \frac{\partial}{\partial x^1} + v^2 \frac{\partial}{\partial x^2} \\
D_Z &= w^1 \frac{\partial}{\partial x^1} + w^2 \frac{\partial}{\partial x^2}
\end{align*}
with $u^i, v^i, w^i \in C^\infty(\mathbb R^2)$. Let $u^1(x^1, x^2) = x^2$ and $w^2(x^1,x^2) = x^1$ as well as $u^2 = v^1 = v^2 = w^1 = 0$.
The commutators are then
\begin{align*}
[X, Y] &= [X, 0] = 0 \\
[Y, Z] &= [0, Z] = 0 \\
[X, Z] &= \sum_{j=1}^2 \sum_{i=1}^2 \left(u^i \frac{\partial w^j}{\partial x^i} - w^i \frac{\partial u^j}{\partial x^i}\right) \frac{\partial}{\partial x^j} \\
&= x^2 \frac{\partial}{\partial x^2} - x^1 \frac{\partial}{\partial x^1} \neq 0
\end{align*}
Letting $Y = 0$ might be cheesy, but it is technically legal, the best kind of legal.\qed

\item Let $t$ be said linear map:
\begin{align*}
t:~ &\mathbb R^{V^*\times V} \to \mathbb R \\
&\sum_i a_i (l_i,v_i) \mapsto \sum_i a_i \langle l_i,v_i\rangle
\end{align*}
Note: on the left is a formal linear combination of basis vectors of $\mathbb R^{V^*\times V}$, on the right is a plain sum in $\mathbb R$.
\begin{enumerate}
\item $\Omega$ is defined to be the linear hull of four types of elements. Since $t$ is linear, it suffices to show that these elements vanish under $t$.
\begin{align*}
t\left(\left(l_1 + l_2, v\right) - \left(l_1, v\right) - \left(l_2, v\right)\right) &=
\langle l_1 + l_2, v\rangle - \langle l_1, v\rangle - \langle l_2, v\rangle \\
&= \langle 0, v \rangle = 0 \\
t\left(\left(l, v_1 + v_2\right) - \left(l, v_1\right) - \left(l, v_2\right)\right) &=
\langle l, v_1 + v_2\rangle - \langle l, v_1\rangle - \langle l, v_2\rangle \\
&= \langle l, 0 \rangle = 0 \\
t\left( \left( \lambda l, v\right) - \lambda \left( l, v\right) \right) &=
\langle \lambda l, v \rangle - \lambda \langle l, v \rangle \\
&= (\lambda - \lambda) \cdot \langle l, v \rangle = 0 \\
t\left( \left( l, \lambda v\right) - \lambda \left( l, v\right) \right) &=
\langle l, \lambda v \rangle - \lambda \langle l, v \rangle \\
&= (\lambda - \lambda) \cdot \langle l, v \rangle = 0 \\
\end{align*}
It all essentially follows from the bilinearity of the canonical pairing. The universal property of the Tensor Product now guarantees us that the unique existence of a function $\tilde t =: tr$:
\begin{align*}
tr:~ V^* &\otimes V \to \mathbb R \\
l &\otimes v \mapsto \langle l, v \rangle
\end{align*}

\item Just calculate:
\begin{align*}
tr(A) &= tr(a_j^i e_i \otimes \eta^j) \\
&= a_j^i tr(e_i \otimes \eta^j) &\text{by linearity of }tr \\
&= a_j^i \langle e_i, \eta^j \rangle &\text{by definition of }tr \\
&= a_j^i \delta_i^j &\text{by definition of dual basis} \\
&= a_i^i &\text{by summation along the diagonal}
\end{align*}
\qed
\end{enumerate}

\item Hilbert's third problem as an exercise, nice. 

\end{enumerate}


\end{document}