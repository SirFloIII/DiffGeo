\documentclass[a4paper,11pt,notitlepage,fullpage]{article}
%\documentclass{report}

\usepackage{fullpage}
\usepackage[utf8]{inputenc}
%\usepackage[ngerman]{babel}
\usepackage[english]{babel}
\usepackage{amsmath}
\usepackage{amssymb}
\usepackage{latexsym}
\usepackage{mathtools}
\usepackage{listings}
\usepackage{algorithm}
\usepackage{algpseudocode}
\usepackage{graphicx}
\usepackage{booktabs}
\usepackage{hhline}
\usepackage{amsthm}
\usepackage{cite}
\usepackage{wrapfig}
\usepackage{hyperref}
\usepackage{titling}
\usepackage{color}
\usepackage[normalem]{ulem}

\setlength{\droptitle}{-60pt}


\newcommand{\y}{\gamma}
\newcommand{\dy}{\dot\gamma}
\newcommand{\ddy}{\ddot\gamma}

\newcommand{\la}{\left\langle}
\newcommand{\ra}{\right\rangle}
\newcommand{\EFFG}{\left(\begin{matrix}E&F\\F&G\end{matrix}\right)}
\newcommand{\LMMN}{\left(\begin{matrix}L&M\\M&N\end{matrix}\right)}

\begin{document}
\author{Florian Bogner}
\title{Differential Geometry - Exercise 1}
\maketitle


\begin{enumerate}
\item \emph{Let $X$ and $Y$ be tangent vector fields on a smooth surface $M \in \mathbb R^3$. Show that the vector field $D_X Y - D_Y X$ along M is also a tangent vector field.}

Let $N$ be a unit normal field. Since $X$ and $Y$ are tangent fields, we have for every point $p$ that $\left\langle X, N \right\rangle = \left\langle Y, N \right\rangle = 0$. Thus also the directional derivative of these terms is $0$. By the Leibniz rule we have:
\begin{align*}
0 = D_X\left\langle Y, N \right\rangle &= \la D_X Y, N \ra + \la Y, D_X N \ra \\
0 = D_Y\left\langle X, N \right\rangle &= \la D_Y X, N \ra + \la X, D_Y N \ra
\end{align*}
Subtracting the two equation from each other and rearranging gives us:
\begin{align*}
\la D_X Y - D_Y X, Z \ra = \la X, D_Y N \ra - \la Y, D_X N \ra
\end{align*}
We want the left side to be $0$, as then the vector field in question would be orthogonal to the unit normal, thus proving it to be tangent. Luckily Theorem 3.11. tells us the right side is $0$. \qed

\item \emph{Consider the surface}
\begin{equation*}
\sigma(u,v) = (u - \frac{u^3}{3} + uv^2, v - \frac{v^3}{3} + u^2v, u^2 - v^2)
\end{equation*}
\begin{enumerate}
\item \emph{Show that the surface is conformally parametrized.}
\item \emph{Show that it has zero mean curvature (do only as much computation as needed.}
\end{enumerate}

For (\emph a) Theorem 2.45. tells us to look at the first fundamental form and see if it is a multiple of the identity matrix. For (\emph b) we use Lemma 3.18. Again we use \emph{sympy} \sout{because we are lazy} to calculate the first and second fundamental form. The source code is on the last page.
\begin{align*}
E &= \left(u^{2} + v^{2} + 1\right)^{2} \\
F &= 0 \\
G &= \left(u^{2} + v^{2} + 1\right)^{2} \\
L &= 2 \\
M &= 0 \\
N &= -2 \\
\end{align*}
Indeed the first fundamental form is a multiple of the identity matrix since $E = G$ and $F = 0$. Thus $\sigma$ is conformally parametrized.\qed

The mean curvature $H$ is
\begin{align*}
H &= \frac{\det\EFFG, \LMMN}{\det\EFFG} \\
&= \frac{\frac{1}{2}\left(EN + GL - FM - FM\right)}{EG - F^2} \\
&= \frac{\frac{1}{2}\left(-2E+2E-0-0\right)}{EG-F^2} \\
&= 0
\end{align*} 
\qed


\item \emph{A vector is said to be \emph{asymtotic} if $II(X,X) = 0$.}
\begin{enumerate}
\item \emph{Show that if the surface $M$ contains the line $p + tX$, then the vector X is asymptotic.}

By definition and by Leibnitz rule we have:
\begin{align*}
II(X,X) &= -\la D_X N, X \ra \\
&= - D_X \la N,X\ra + \la N, D_X X\ra \\
&= - D_X 0 + \la N, 0 \ra \\
&= 0
\end{align*}
For the directional derivatives let $\gamma(t) = p + tX$. The first summand is 0 because the unit normal vector field $N$ is, well, normal to vectors inside the surface such as $X$. The second summand is 0 as $D_X X$ is zero because $X$ is constant with respect to $t$. The direction of a line is always the same. \qed

\item \emph{Assume that the second fundamental form is non-degenerate and indefinite at p, so that it has two different asymptotic directions. Show that these directions are orthogonal if and only if the mean curvature vanishes at p.}

Let $X$ and $Y$ be to two asymptotic vectors representing the two different directions. Since they are asymptotic, i.e. $\la S(X), X \ra = 0$, we have that $(X, S(X))$ is a (orthogonal) basis. Lets write the shape operator in this basis:
\begin{equation*}
S = \left(\begin{matrix}0&a\\1&b\end{matrix}\right)
\end{equation*}
The characteristic polynomial is $t^2 - bt - a = 0$. Its roots are the principal curvatures. The mean curvature is zero iff the roots are symmetric around $0$ which is the case iff $b = 0$ which is equivalent to $S(Y) = aX$. Since $S(Y) \perp Y$, this is the case iff $X \perp Y$.\qed
\end{enumerate}


\item \emph{Show that the tangent developable of every curve has Gauss curvature zero.}

Let $\gamma$ be a curve. Its tangent developable is $\sigma(u,v) = \gamma(u) + v\cdot\dot\gamma(u)$. For brevitiy we will omit arguments of $u$. To use Lemma 3.18. we calculate (parts of) the second fundamental form:
\begin{align*}
\sigma_u &= \dy + v\ddy \\
\sigma_v &= \dy \\
\sigma_{uu} &= \text{unimportant} \\
\sigma_{uv} &= \ddy \\
\sigma_{vv} &= 0 \\
\mathcal N = \frac{\sigma_u \times \sigma_v}{\|\sigma_u \times \sigma_v\|} &= c\cdot(\dy\times\ddy) ~\text{ for some } c\\
L = \la\sigma_{uu}, \mathcal N \ra &= \text{unimportant} \\
M = \la\sigma_{uu}, \mathcal N \ra &= c\cdot \la\ddy, \dy\times\ddy\ra = 0 \\
N = \la\sigma_{uu}, \mathcal N \ra &= 0 \\
\end{align*}
Thus the Gauss curvature is
\begin{equation*}
K = \frac{\det\left(\begin{matrix}L&0\\0&0\end{matrix}\right)}{\det I} = 0
\end{equation*}
\qed

\end{enumerate}

\newpage

\begin{verbatim}
import sympy as sp
from sympy import diff, sqrt

class Vec3:    
    def __init__(self, x, y, z):
        self.x = x
        self.y = y
        self.z = z
        
    def __add__(self, other):
        return Vec3(self.x + other.x,
                    self.y + other.y,
                    self.z + other.z)
        
    def __mul__(self, scalar):
        return Vec3(scalar * self.x,
                    scalar * self.y,
                    scalar * self.z)
    
    __rmul__ = __mul__
    
    def __truediv__(self, scalar):
        return Vec3(self.x / scalar,
                    self.y / scalar,
                    self.z / scalar)
    
    def diff(self, var):
        return Vec3(diff(self.x, var),
                    diff(self.y, var),
                    diff(self.z, var))
        
    def dot(self, other):
        return self.x * other.x + self.y * other.y + self.z * other.z
    
    def cross(self, other):
        return Vec3(self.y * other.z - self.z * other.y,
                    self.z * other.x - self.x * other.z,
                    self.x * other.y - self.y * other.x)
        
    def norm(self):
        return sqrt(self.x**2 + self.y**2 + self.z**2)

sp.init_printing(forecolor = 'White')
u, v = sp.symbols("u v", real = True)

sig = Vec3(u - u**3/3 + u * v**2,
           v - v**3/3 + u**2 * v,
           u**2 - v**2)

sig_u = sig.diff(u)
sig_v = sig.diff(v)

sig_uu = sig.diff(u).diff(u)
sig_uv = sig.diff(u).diff(v)
sig_vv = sig.diff(v).diff(v)

c = sig_u.cross(sig_v)

NN = c / c.norm()

E = sig_u.dot(sig_u)
F = sig_u.dot(sig_v)
G = sig_v.dot(sig_v)

L = sig_uu.dot(NN)
M = sig_uv.dot(NN)
N = sig_vv.dot(NN)

print("\\begin{align*}")
for l in "EFGLMN":
    print(l,"&=",sp.latex(sp.simplify(sp.factor(sp.expand(sp.simplify(eval(l)))))), "\\\\")
print("\\end{align*}")
\end{verbatim}

\end{document}